%%%%%%%%%%%%%%%%%
% This CV created using altacv.cls
% (v1.6.4, 13 Nov 2021) written by LianTze Lim (liantze@gmail.com). Now compiles with pdfLaTeX, XeLaTeX and LuaLaTeX.
%
%% It may be distributed and/or modified under the
%% conditions of the LaTeX Project Public License, either version 1.3
%% of this license or (at your option) any later version.
%% The latest version of this license is in
%%    http://www.latex-project.org/lppl.txt
%% and version 1.3 or later is part of all distributions of LaTeX
%% version 2003/12/01 or later.
%%%%%%%%%%%%%%%%

%% Use the "normalphoto" option instead of "ragged2e,withhyper" if you want a normal photo instead of cropped to a circle
\documentclass[10pt,a4paper,ragged2e,withhyper]{altacv}
%% AltaCV uses the fontawesome5 and packages.
%% See http://texdoc.net/pkg/fontawesome5 for full list of symbols.

% Change the page layout if you need to
\geometry{left=1.25cm,right=1.25cm,top=1.5cm,bottom=1.5cm,columnsep=0.8cm}

% The paracol package lets you typeset columns of text in parallel
\usepackage{paracol}

% Change the font if you want to, depending on whether
% you're using pdflatex or xelatex/lualatex

% Change the colours if you want to
% \definecolor{SlateGrey}{HTML}{2E2E2E}
% \definecolor{LightGrey}{HTML}{666666}
% \definecolor{DarkPastelRed}{HTML}{450808}
% \definecolor{PastelRed}{HTML}{8F0D0D}
% \definecolor{GoldenEarth}{HTML}{E7D192}
\colorlet{name}{black}
\colorlet{tagline}{black}
\colorlet{heading}{black}
\colorlet{headingrule}{black}
\colorlet{subheading}{black}
\colorlet{accent}{black}
\colorlet{emphasis}{black}
\colorlet{body}{black}

% Change some fonts, if necessary
\renewcommand{\namefont}{\Huge\rmfamily\bfseries}
\renewcommand{\personalinfofont}{\footnotesize}
\renewcommand{\cvsectionfont}{\LARGE\rmfamily\bfseries}
\renewcommand{\cvsubsectionfont}{\large\bfseries}


% Change the bullets for itemize and rating marker
% for \cvskill if you want to
\renewcommand{\itemmarker}{{\small\textbullet}}
\renewcommand{\ratingmarker}{\faCircle}

\begin{document}
\name{Elizaveta Kotelnikova}
\tagline{Java/Kotlin backend developer}

%% You can add multiple photos on the left or right
% \photoR{2.8cm}{Globe_High}
% \photoL{2.5cm}{Yacht_High,Suitcase_High}

\personalinfo{
  \location{St. Petersburg, Russia}
%   \mailaddress{Åddrésş, Street, 00000 Cóuntry}
  \email{elizaveta1408li@gmail.com}
%\linkedin{elizavetakotelnikova}
  \github{elizavetakotelnikova}
  \phone{+79816999984}
  \telegram{tg: @kotelnikova14}
  %% You can add your own arbitrary detail with
  %% \printinfo{symbol}{detail}[optional hyperlink prefix]
  % \printinfo{\faPaw}{Hey ho!}[https://example.com/]
  %% Or you can declare your own field with
  %% \NewInfoFiled{fieldname}{symbol}[optional hyperlink prefix] and use it:
  % \NewInfoField{gitlab}{\faGitlab}[https://gitlab.com/]
  % \gitlab{your_id}
  %%
  %% For services and platforms like Mastodon where there isn't a
  %% straightforward relation between the user ID/nickname and the hyperlink,
  %% you can use \printinfo directly e.g.
  % \printinfo{\faMastodon}{@username@instace}[https://instance.url/@username]
}

\makecvheader
%% Depending on your tastes, you may want to make fonts of itemize environments slightly smaller
% \AtBeginEnvironment{itemize}{\small}

%% Set the left/right column width ratio to 6:4.
\columnratio{0.7}

% Start a 2-column paracol. Both the left and right columns will automatically
% break across pages if things get too long.
\begin{paracol}{2}
\cvsection{Work Experience}

\cvevent{Kotlin Backend developer}{T-Bank}{Oct 2024 -- Ongoing}{Saint Petersburg, Russia}

\begin{itemize}
  \item Implemented backend for multiple new services, now actively used by bank employees, boosting work efficiency, measurability, and overall convenience
  \item Migrated services' data to a centralized Data Warehouse (DWH), improving system reliability and enabling analytics of business processes
  \item Established microservice communication and developed API integrations with existing bank infrastructure, ensuring interoperability and incorporation of newly developed services
  \item Achieved ~70\% code coverage through integration and unit testing, reducing system defects and enhancing overall stability
\end{itemize}

\cvsection{Related Experience}
%\cvevent{}{Dell Technologies}{}{St. Petersburg, Russia}
\cvevent{Algorithms and data structures TA}{ITMO University}{Sep 2023 - Jan 2025}{St. Petersburg, Russia}

\begin{itemize}
  \item Designed and implemented algorithmic tasks/solutions using C++
  \item Organized and guided multiple student groups, ensuring a structured and effective learning process
\end{itemize}

\divider

\cvevent{Backend Developer}{"VK \& ITMO Mini-apps" Hackathon}{Oct 2024}{St. Petersburg, Russia}
\begin{itemize}
  \item In a team of three, built and deployed backend for a university application
  \item Designed and implemented API to enable frontend interaction
\end{itemize}

\divider

\cvevent{Backend Developer}{"NoLabel Competiton" Hackathon}{Jan 2024}{St. Petersburg, Russia}

\textit{Prize in the "Best MVP" category}
\begin{itemize}
  \item Developed RESTful webservice in the team of 3 backend and 2 frontend developers
  \item Designed database structure, used PostgreSQL to manage data, implemented Data Access Layer
  \item Collaborated on business logic development
  \item Worked on technical specification, ensuring all usecases are well-documented
  \item Achieved high test coverage from writing unit and integration tests
\end{itemize}

\github{elizavetakotelnikova/Let-s-lunch-app}

\divider
\cvevent{}{Yandex IT TA Competition}{}{}
Winner of the IT Teaching Assistant Competition 2024

%% Switch to the right column. This will now automatically move to the second
%% page if the content is too long.
\switchcolumn

\cvsection{Education}
% \cvevent{\normalsize{BSc, Software Engineering} \footnotesize{(GPA 3.3)}}{ITMO University}{Sept 2022 -- June 2026}{}
\cvevent{\normalsize{BSc, Software Engineering}\footnotesize{ (GPA 3.8/4)}}{ITMO University}{Sept 2022 -- June 2026}{}

\cvsection{Skills}

\cvevent{Programming Languages}{}{}{}

\cvtag{Kotlin}
\cvtag{Java}
\cvtag{Go}
\cvtag{SQL}

% \cvtag{JavaScript/TypeScript}

\medskip

\cvevent{Technologies}{}{}{}

\cvtag{Spring Framework}
\cvtag{Kafka}
\cvtag{RabbitMQ}
\cvtag{PostgreSQL}

\cvtag{MongoDB}

\medskip

\cvevent{Tools}{}{}{}

\cvtag{Docker}
%\cvtag{Kubernetes}
\cvtag{Linux}
\cvtag{Git}

\cvtag{Prometheus}
\cvtag{Grafana}

\cvtag{GitLab CI/CD}
%\cvtag{Jenkins}


\cvsection{Languages}

\begin{itemize}
\item English - advanced (C1)
\item German - intermediate
\item Russian - native
\end{itemize}

\cvsection{Projects}

%\cvevent{Expenses Calculator}{}{}{}

%Console application designed to help users track information about their %expenses based on categories or date range (Java)

%\github{elizavetakotelnikova/ExpensesCalculator}

%\divider

\cvevent{UserInfoService}{}{}{}

Spring-powered web-application that manages users' information (Java, Spring Framework, PostgreSQL)

\github{elizavetakotelnikova/UserInfoService}

%%\divider

%% \cvevent{Custom programming language}{}{}{}

%% In a team of four, implemented a custom programming language: Parser, %% Lexer, JIT and GC (C++, LLVM)

%%\github{elizavetakotelnikova/own-prog-lang}

%%\cvevent{ExpensesCalculator}{}{}{}

%%Console application to summarize expenses by month/group/code. Used %%algorithms and data structures alongside OOP patterns to provide command %%handling

%%\github{elizavetakotelnikova}

%% Yeah I didn't spend too much time making all the
%% spacing consistent... sorry. Use \smallskip, \medskip,
%% \bigskip, \vspace etc to make adjustments.

\end{paracol}

\end{document}
